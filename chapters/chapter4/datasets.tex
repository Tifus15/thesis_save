\chapter{Datasets}
In this Chapter we will discuss some Physics cases which totally obey conservative property of total energy(Hamiltonian). In previous chapter we discused about oscilator which is trival model and how we found a solution of the dynamics of the model. In this chapter we will not only show how oscilator dataset should be created but we will discuss about 4 models which are:
\begin{itemize}
	\item  N body problem
	\item two body problem
	\item threebody problem
	\item N pendelum
\end{itemize}

\section{Oscilator}
As we already shown the oscillator solution we could show solution of the oscilator trough hamiltonian equations.
For this we will use Kinetic Energy and Potentional energy $U = \frac{1}{2}kx^2$ 
Kinetic energy is defined as $T =\frac{1}{2}mv^2$ or if we use momentum $p = mv$ we get  $T =\frac{1}{2m}p^2$.
Our Hamiltonian will look like:
\begin{equation}
	\mathcal{H} = \frac{1}{2m}p^2 + \frac{1}{2}kx^2
\end{equation} 
This case is trivial.
We could get $\dot{x}$ and $\dot{p}$ trough hamiltonian equations $\dot{q} = \frac{\partial H}{\partial p}$ and $\dot{p} = - \frac{\partial H}{\partial q}$
Or we could compare Hamiltonian with elipse formula
\begin{equation}
	1 = \frac{x^2}{a^2} + \frac{y^2}{b^2}
\end{equation} and set the components in parametric equation\begin{equation}
(x,y) = (a\cos(\phi),b\sin(\phi)) \rightarrow(p,q) = \left(\sqrt{2Hm}\cos(\phi),\sqrt{\frac{2H}{k}}\sin(\phi)\right)
\end{equation} of the ellipse. This equation represents kinematic of the oscilator. To find periodicity let say that $\phi = \omega t + \Phi $ and we know that the $p = mx$, with this equation we can calculate $\omega$.
\begin{eqnarray}
	p &=& m\dot{x} =\sqrt{2Hm}\cos(\omega t + \Phi) \\
	\dot{x} &=& \sqrt{\frac{2H}{m}}\cos(\omega t + \Phi)\\
		x &=& \int \sqrt{\frac{2H}{m}}\cos(\omega t + \Phi)dt = \frac{1}{\omega}\sqrt{\frac{2H}{m}}\sin(\omega t + \Phi) = \sqrt{\frac{2H}{k}} \sin(\omega t + \Phi)\\
	\omega &=& \sqrt{\frac{k}{m}} 
\end{eqnarray}  




The oscliator has harmonic, periodic movement which means that after some time $T$ the movement will be repeated. With that property we can calculate the period over circle circumference:
\begin{equation}
	T = \frac{2\pi}{\omega}
\end{equation}

With this we can create a dataset within the program code. For coding we used framework pytorch to calculate this dataset numerically.\\
For the obtain the solution of ODE we used package torchdiffeq to solve numerical equation with dopri5 method which is most accurate method for solving ODEs.\\
To create unique trajectories for the dataset we just need to specify energy region.


\section{N-body problem}
N-body problem is most complex problem in physics. It describes a movement of bodies in the free space. There are no conditions and there are multitude of solutions, because this problem if it isn't deterministic it goes under the quantum theory. 
In Newtonian formulation the N-body problem is
\begin{equation}
m_i\ddot{\mathbf{x}}_i = -\sum_{i\neq j, j=1}^nG\frac{m_im_j(\mathbf{x}_i-\mathbf{x}_j)}{||\mathbf{x}_i-\mathbf{x}_j||^3}
\end{equation} 
and its Hamiltonian:
\begin{equation}
	\mathcal{H} =\sum_i^n \frac{1}{2m_i}||\mathbf{p}_i||^2 - \sum_{1<i,j<n,i\neq j}\frac{Gm_im_j}{||\mathbf{x}_i-\mathbf{x}_j||}
\end{equation} 
for $i \in[1,n].$
In our use case to create  dataset we needed to make some conditions that our trajectories have if possible constant hamiltonian.\\
The biggest issue in this dataset was the crashes between the bodies.
This can be resolved with simple trick adding the $\epsilon$ to fix maximal potentional energy
\begin{equation}
	- \sum_{1<i,j<n,i\neq j}\frac{Gm_im_j}{||\mathbf{x}_i-\mathbf{x}_j||+\epsilon}.
\end{equation}
We secured that the value in dominator is never 0. This is important because we use numerical solvers to create trajectories. If we have energy which achieves very big difference in values in one time step, our solution will be inaccurate. For calculation of such trajectories we are forced not to make Hamiltonian equations, but to calculate the trajectories using acceleration, velocity and position of the bodies trough some symplectic numerical methods.
\subsection{Symplectic numerical methods to solve N-body problem}
Symplectic numerical methods are designed for numerical solution of Hamilton Equations. It posses conserved quality to approximate the Total Energy the Hamiltonian.\\
They have two Forms. One in canonical coordinates(q,p) and in langrangian coordinates(x,v,a). Most used ones are in langrangian form and we will use it to create trajectory for N-body problem.\\
For the N-body problem we used two numerical solvers. In first step we used velocity verlet
\begin{eqnarray}
	\mathbf{x}_{i+1} &=& \mathbf{x}_i + h\mathbf{v}_i + \mathbf{a}_i\frac{h^2}{2} \\
	\mathbf{v}_{i+1}&=&\mathbf{v}_i + (\mathbf{a}_i +\mathbf{a}_{i+1})\frac{h}{2}
\end{eqnarray} and
for the next steps with timestep $h$ we used more stable beeman method
\begin{eqnarray}
	\mathbf{x}_{i+1} &=& \mathbf{x}_i + h\mathbf{v}_i + \frac{h^2}{6}(4\mathbf{a}_{i}-\mathbf{a}_{i-1})\\
	\mathbf{v}_{i+1}&=&\mathbf{v}_i + \frac{h}{6}(2\mathbf{a}_{i+1} + 5\mathbf{a}_i-\mathbf{a}_{i+1}).
\end{eqnarray}
This needed to be done because the beeman method needs acceleration of the next and previous step to be calculated.
\subsection{twobody problem}
Two body problem is one of the most easiest dataset which could be created from N-body problem. It has one trivial solution which is Binary star movement.
Two stars trough their movemnt maintain constant distance $r$ and their center of mass $\mathbf{p}_m$ is static at the origin.\\
The newtonian form of twobody Problem is written as
\begin{eqnarray}
	m_1\ddot{\mathbf{q}}_1 = -G\frac{m_1m_2(\mathbf{q}_1-\mathbf{q}_2)}{||\mathbf{q}_1-\mathbf{q}_2||^3}\\
	m_2\ddot{\mathbf{x}}_2 = -G\frac{m_1m_2(\mathbf{q}_2-\mathbf{q}_1)}{||\mathbf{q}_2-\mathbf{q}_1||^3}
\end{eqnarray}
From those equations we can get a statement about third newton law:
\begin{equation}
	m_1\ddot{\mathbf{q}}_1 -m_2\ddot{\mathbf{q}}_2 = 0 = \mathbf{F}_{ij} - \mathbf{F}_{ji}
\end{equation}
and its hamiltonian
\begin{equation}
\mathcal{H} = \frac{||\mathbf{p}_1||}{2m_1} +\frac{||\mathbf{p}_2||}{2m_2} - G\frac{m_1m_2}{||\mathbf{q}_1 - \mathbf{q}_2||}
\end{equation}
We said that we have static center mass point. center mass point is obtained using following formula
\begin{equation}
	\mathbf{q}_m = \frac{1}{m_1+m_2}(m_1\mathbf{q_1} + m_2\mathbf{q_2})
\end{equation}
if we set the center mass point in origin the equation simplifies nd we get one about velocity
\begin{eqnarray}
	m_1\mathbf{q_1} + m_2\mathbf{q_2} = 0\\
	m_1\dot{\mathbf{q}}_1 + m_2\dot{\mathbf{q}}_2 = \mathbf{p}_1 +\mathbf{p}_2 = 0
\end{eqnarray}.
From now on we will write $\mathbf{p} =\mathbf{p}_1 = -\mathbf{p}_2$.
Lets make some formulas about distance. 
\begin{eqnarray}
	\mathbf{q}_2 = \frac{m_1}{m_2}\mathbf{q}_2\\
	||\mathbf{q}_2|| =^{\mathbf{q}_m=0} \frac{m_1}{m_2}||\mathbf{q}_1||\\ 
	r_2 =\frac{m_1}{m_2}r_1
\end{eqnarray}
With this we got interesting property
\begin{equation}
	\frac{r_1}{r_2} =\frac{m_2}{m_1} 
\end{equation}.
This can be used to calculate constant distance between the bodies
\begin{equation}
	r=r_1\left(1 + \frac{m_1}{m_2}\right)
\end{equation}

With those equations we can simplify hamiltonian \begin{equation}
	\mathcal{H}= ||\mathbf{p}||^2\left(\frac{1}{m_1}+\frac{1}{m_2}\right) - G\frac{m_1m_2}{r}
\end{equation}.
Still to make a dataset with twobody trajectories we need a period $T$. Let assume that our solution as describe is radial. the bodies moves on a circular path
\begin{eqnarray}
	\mathbf{q}_1 = r_1
	\begin{bmatrix}
		\cos(\Phi)\\
		\sin(\Phi)
	\end{bmatrix}\\
\mathbf{q}_2 = r_2
\begin{bmatrix}
	\cos(\Phi+ \pi)\\
	\sin(\Phi+ \pi)
\end{bmatrix}
\end{eqnarray}
The momentum is tangential in dependency of orientation of position
\begin{eqnarray}
	\mathbf{p}_1 = ||\mathbf{p}||
	\begin{bmatrix}
		\sin(\Phi)\\
		-\cos(\Phi)
	\end{bmatrix}\\
	\mathbf{p}_2 = ||\mathbf{p}||
	\begin{bmatrix}
		\sin(\Phi+ \pi)\\
		-\cos(\Phi+ \pi)
	\end{bmatrix}
\end{eqnarray} with $\Phi \in [0,2\pi] $.
Let say that $\Phi=\omega t$
and we want to find a period of movement.
With the equations
\begin{eqnarray}
	\dot{\mathbf{q}}_1 = r_1\omega
	\begin{bmatrix}
		\sin(\omega t)\\
		-\cos(\omega t)
	\end{bmatrix}\\ = \frac{\mathbf{p_1}}{m_1} = \frac{||\mathbf{p}||}{m_1}\begin{bmatrix}
	\sin(\omega t)\\
	-\cos(\omega t)
\end{bmatrix}\\
||\mathbf{p}|| = r_1m_1\omega
\end{eqnarray},
we calculated the value of the momentum.
This momentum we can get from Physics and newton third law. If some body moves around some point in radial path, we can calculate a centripetal Force and this Force should be equal to Gravitational Force
\begin{equation}
	F_c = \frac{||\mathbf{p}||^2}{m_1r_1} = \frac{Gm_1m_2}{r^2} = F_g
\end{equation}
Substituting $||\mathbf{p}||$ we get $\omega$.
\begin{equation}
	\omega = \frac{1}{r}\sqrt{G\frac{m_2}{r_1}}
\end{equation}
In two body problem to create some diverse dataset but with equivalent bodies we vary the distance between the bodies. We need to be careful because with grater distance $r$ the period $T$ will be longer.\\
The mathematical formulation of the solution which is used for code is taken from hamiltonian equations
\begin{equation}
	\dot{\mathbf{x}} = J\frac{\partial H}{\partial \mathbf{x}}(\mathbf{x})=
\begin{bmatrix}
	0 & 0 & 0 & 0 & \frac{1}{m_1} & 0\\
	0 & 0 & 0 & 0 & 0 & \frac{1}{m_1}\\
	0 & 0 & 0 & 0 & -\frac{1}{m_2} & 0\\
	0 & 0 & 0 & 0 & 0 & -\frac{1}{m_2}\\
	-\mu_{1,2} & 0 & \mu_{1,2} & 0 & 0 & 0 \\
	0 & -\mu_{1,2} & 0 & \mu_{1,2} & 0 & 0 \\
\end{bmatrix}
\begin{bmatrix}
	q_{1x}\\
	q_{1y}\\
	q_{2x}\\
	q_{2y}\\
	p_x\\
	p_y
\end{bmatrix} 
\end{equation} where $\mu_{ij}=\frac{Gm_im_j}{r^3}$ 

\subsection{Three body problem}
Three body problem is from formulation similar to two body problem, but finding the periodic solutions is very complex task. Fortunatly the M. Šuvlakov and V. Dmitrašinović have found many of the initial values to create periodic three body trajectory and made data public over the website for everyone to use.\cite{papthreebody}\cite{web}. But under some definitions such like the gravitational constant $G$ and all masses $m$ are 1.
Even today there are scientists which are documenting initial values for more periodic solutions \cite{hudomal2015new}.
Having those conditions we need just Hamiltonian equations which are easy to obtain. 

Obtaining the formula for straight-forward calculation the trajectories of the threebody problem we need to define a function
\begin{equation}
	\boldsymbol{\Xi}(\mathbf{q}_i,\mathbf{q}_j)=\boldsymbol{\Xi}_{i,j}=\frac{Gm_im_j}{||(\mathbf{q}_i-\mathbf{q}_j)||^3}.
\end{equation}
In following consider $\mathbf{m_i^{⁻1}}=\text{diag}(\frac{1}{m_i},\frac{1}{m_i})$ and coordinates are two dimensional.
\begin{equation}
	\dot{\mathbf{x}} = 
	\begin{bmatrix}
		0 & 0 & 0 & \mathbf{m_1^{⁻1}} & 0 & 0\\
		0 & 0 & 0 & 0 & \mathbf{m_2^{⁻1}} & 0\\
		0 & 0 & 0 & 0 & 0 & \mathbf{m_3^{⁻1}}\\
		-(\boldsymbol{\Xi}_{1,2}+\boldsymbol{\Xi}_{1,3}) & \boldsymbol{\Xi}_{1,2} & \boldsymbol{\Xi}_{1,3} & 0 & 0 & 0\\
		\boldsymbol{\Xi}_{1,2} & -(\boldsymbol{\Xi}_{1,2}+\boldsymbol{\Xi}_{2,3}) & \boldsymbol{\Xi}_{2,3} & 0 & 0 & 0\\
		\boldsymbol{\Xi}_{1,3} & \boldsymbol{\Xi}_{2,3} & -(\boldsymbol{\Xi}_{1,3}+\boldsymbol{\Xi}_{2,3}) & 0 & 0 & 0\\
	\end{bmatrix}
	\begin{bmatrix}
		\mathbf{q}_1\\
		\mathbf{q}_2\\
		\mathbf{q}_3\\
		\mathbf{p}_1\\
		\mathbf{p}_2\\
		\mathbf{p}_3
	\end{bmatrix}
\end{equation}
The three-body problem have three interesting properties due its defined potentional energy. Their potential energy is defined over gravitational force between the bodies. In another words we can translate the trajectory in space and rotate it. Let us prove this over Hamiltonian Energy
\begin{equation}
	\mathcal{H} = \frac{||\mathbf{p}_1||^2}{2m_1} +\frac{||\mathbf{p}_2||^2}{2m_2}+\frac{||\mathbf{p}_3||^2}{2m_3} - G\frac{m_1m_2}{||\mathbf{q}_1 - \mathbf{q}_2||}-G\frac{m_2m_3}{||\mathbf{q}_2 - \mathbf{q}_3||}-G\frac{m_1m_3}{||\mathbf{q}_1 - \mathbf{q}_3||} 
\end{equation}-
\begin{itemize}
	\item Translation:\\
	Lets define $\mathbf{q}_i = \lim_{\mathbf{q}_t\rightarrow \mathbf{0}}(\mathbf{q}_i + \mathbf{q}_t)$ where $\mathbf{q}_t$ is translation vector. When we substitute $\mathbf{q}_i$. The hamiltoinian acts as follows
	\begin{eqnarray*}
		\mathcal{H} &=& \frac{||\mathbf{p}_1||^2}{2m_1} +\frac{||\mathbf{p}_2||^2}{2m_2}+\frac{||\mathbf{p}_3||^2 }{2m_3}\\
		& & - \lim_{\mathbf{q}_t\rightarrow \mathbf{0}}G\frac{m_1m_2}{||\mathbf{q}_1+ \mathbf{q}_t  - (\mathbf{q}_2+ \mathbf{q}_t) ||}-\lim_{\mathbf{q}_t\rightarrow \mathbf{0}}G\frac{m_2m_3}{||\mathbf{q}_2 + \mathbf{q}_t - (\mathbf{q}_3+ \mathbf{q}_t) ||}\\
		& &-\lim_{\mathbf{q}_t\rightarrow \mathbf{0}}G\frac{m_1m_3}{||\mathbf{q}_1+ \mathbf{q}_t  - (\mathbf{q}_3+ \mathbf{q}_t) ||}\\
		&=& \frac{||\mathbf{p}_1||}{2m_1} +\frac{||\mathbf{p}_2||}{2m_2}+\frac{||\mathbf{p}_3||}{2m_3}\\ & &-  G\frac{m_1m_2}{||\mathbf{q}_1 - \mathbf{q}_2 +(\mathbf{q}_t-\mathbf{q}_t) ||}-G\frac{m_2m_3}{||\mathbf{q}_2 - \mathbf{q}_3 +(\mathbf{q}_t-\mathbf{q}_t)||}-G\frac{m_1m_3}{||\mathbf{q}_1 - \mathbf{q}_3+(\mathbf{q}_t-\mathbf{q}_t)||}\\
		& = & \frac{||\mathbf{p}_1||}{2m_1} +\frac{||\mathbf{p}_2||}{2m_2}+\frac{||\mathbf{p}_3||}{2m_3} - G\frac{m_1m_2}{||\mathbf{q}_1 - \mathbf{q}_2||}-G\frac{m_2m_3}{||\mathbf{q}_2 - \mathbf{q}_3||}-G\frac{m_1m_3}{||\mathbf{q}_1 - \mathbf{q}_3||} 
	\end{eqnarray*}
	\item Rotation:
	Lets define $\mathbf{q}_i = r_i \begin{bmatrix}
		\cos(\beta + \alpha)\\
		\sin(\beta + \alpha)\\
		\end{bmatrix}$ and $\mathbf{p}_i = p_i \begin{bmatrix}
		-\sin(\beta + \alpha)\\
		\cos(\beta + \alpha)\\
		\end{bmatrix}$ where $\beta \in (0,2\pi]$ but it is defined and $\alpha \in (0,2\pi] $  which is free parameter.\\
	Let substitute this in Hamiltonian
	\begin{eqnarray*}
		\mathcal{H} &=& \frac{\left|\left| p_1 \begin{bmatrix}
				-\sin(\beta + \alpha)\\
				\cos(\beta + \alpha)\\
			\end{bmatrix}\right|\right|^2}{2m_1} +\frac{\left|\left| p_2 \begin{bmatrix}
			-\sin(\beta + \alpha)\\
			\cos(\beta + \alpha)\\
		\end{bmatrix}\right|\right|^2}{2m_2}+\frac{\left|\left| p_3 \begin{bmatrix}
		-\sin(\beta + \alpha)\\
		\cos(\beta + \alpha)\\
	\end{bmatrix}\right|\right|^2}{2m_3}\\& & - G\frac{m_1m_2}{\left|\left|r_1 \begin{bmatrix}
	\cos(\beta + \alpha)\\
	\sin(\beta + \alpha)\\
\end{bmatrix} - r_2 \begin{bmatrix}
\cos(\beta + \alpha)\\
\sin(\beta + \alpha)\\
\end{bmatrix}\right|\right|}\\
& &-G\frac{m_2m_3}{\left|\left|r_2 \begin{bmatrix}
\cos(\beta + \alpha)\\
\sin(\beta + \alpha)\\
\end{bmatrix} - r_3 \begin{bmatrix}
\cos(\beta + \alpha)\\
\sin(\beta + \alpha)\\
\end{bmatrix}\right|\right|}\\
& &-G\frac{m_1m_3}{\left|\left|r_1 \begin{bmatrix}
\cos(\beta + \alpha)\\
\sin(\beta + \alpha)\\
\end{bmatrix} - r_3 \begin{bmatrix}
\cos(\beta + \alpha)\\
\sin(\beta + \alpha)\\
\end{bmatrix}\right|\right|} 
\end{eqnarray*}
From trigonometry we know that $\cos(\phi)^2 + \sin(\phi)^2= 1 $ which means $\left|\left|\begin{bmatrix}
	\cos(\beta + \alpha)\\
	\sin(\beta + \alpha)\\
\end{bmatrix}\right|\right| =\left|\left|\begin{bmatrix}
\cos(\beta)\\
\sin(\beta)\\
\end{bmatrix}\right|\right|= 1$.\\
Applying this theorem we get 
\begin{equation}
	\mathcal{H} = \frac{p_1^2}{2m_1} +\frac{p_2^2}{2m_2}+\frac{p_3^2}{2m_3} - G\frac{m_1m_2}{|r_1 - r_2|}-G\frac{m_2m_3}{|r_2 - r_3|}-G\frac{m_1m_3}{|r_1 - r_3|}
\end{equation} which is equivalent to the previous definition.
\end{itemize}
For the dataset creation we wanted that our trajectory mass middle point is fixed. We needed only rotation around it.
For such rotation we used formula for such transformation
\begin{equation}
	\hat{\mathbf{q}} = 
	\begin{bmatrix}
		1 & 0 & 0 & x_m\\
		0 & 1 & 0 & y_m\\
		0 & 0 & 1 & z_m\\
		0 & 0 & 0 & 1
	\end{bmatrix}	
\begin{bmatrix}
	\cos(\alpha) & -\sin(\alpha) & 0 & 0\\
	\sin(\alpha) & \cos(\alpha) & 0 & 0\\
	0 & 0 & 1 & 0\\
	0 & 0 & 0 & 1
\end{bmatrix}
\begin{bmatrix}
	1 & 0 & 0 & -x_m\\
	0 & 1 & 0 & -y_m\\
	0 & 0 & 1 & -z_m\\
	0 & 0 & 0 & 1
\end{bmatrix}
\begin{bmatrix}
	x_q\\
	y_q\\
	z_q\\
	1
\end{bmatrix}
\end{equation} which for two dimensional cases can be rewritten as:
\begin{equation}
	\hat{\mathbf{q}} = \mathbf{R}\mathbf{q} - \mathbf{R}\mathbf{q}_m + \mathbf{q}_m
\end{equation} where $\mathbf{R}=\begin{bmatrix}
\cos(\alpha) & -\sin(alpha)\\
\sin(\alpha) & \cos(\alpha)\\
\end{bmatrix}$.

	


 
 \section{N Pendulum}
 The N Pendelum is straight forward model.
 To make it in canocnical coordinates which are angles of the joints, still we need to define Kinematic of the model.
 \begin{eqnarray}
 	x_i = \sum_i^n l_i\sin(\Theta_i)\\
 	y_i = -\sum_i^n l_i\cos(\Theta_i)
 \end{eqnarray}
Their temporal derivations are:
\begin{eqnarray}
\dot{x}_i = \sum_i^n \dot{\Theta}_i l_i\cos(\Theta_i)\\
	\dot{y}_i = \sum_i^n \dot{\Theta}_i l_i\sin(\Theta_i)
 \end{eqnarray}
From this kinematic model we just calculated the Hamiltonian and proceed with autodifferentiaion from pytorch framework to calculate the values of hamiltonian equations
\begin{eqnarray}
	\mathcal{H} = \frac{1}{2}\sum_i^n m_i (\dot{x}_i^2 + \dot{y}_i^2) + \sum_i^n m_ig_iy_i 
\end{eqnarray}.
We set a canonical coordinates $(\Theta, p_{\Theta})$ and we can calculate $p_{\Theta} = \frac{\partial H}{\partial \dot{\Theta}}$.
Making the Hamiltonan only dependent on $\Theta$ and $p_{\Theta}$ we can get hamiltonian equations:
\begin{eqnarray}
	\dot{\Theta_i} = \frac{\partial \mathcal{H}}{\partial p_{\Theta_i}}\\
	\dot{p_{\Theta_i}} = -\frac{\partial \mathcal{H}}{\partial \Theta}
\end{eqnarray}
At the Dataset creation we made randomised dataset with $\Theta \in [-\frac{\pi}{4},\frac{\pi}{4}]$ and starting $p_{\Theta_i} = 0 $.\\
We can create a dataset with observation. With interest to creating of urdf files, we create the framework which trough python coding crates accurate urdf model. This was used to create pendelum with n number of joints. This model was observed in pybullet. From the use case we suggest if you are working with pendeluns where $N<10$ please use the pytorch numerical technique, otherwise use pybullet observation. This is due the numerical stability. More joints or degrees of freedom, hamiltonian even if it should be conservative can't hold stability. In that case using pybullet and decide if Hamiltonian of that trajectory conservative enough trough calculation the mean and standard deviation of the energy sample. 





 
 


 

